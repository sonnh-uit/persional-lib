\begin{preface}{Mở đầu}
 
Trong hàng thập kỷ phát triển của ngành Công nghệ Thông tin, máy tính luôn là một trong những điểm khởi đầu cũng như nút thắt cuối cùng trong sự phát triển của công nghệ. Rào cản về điện tử và vật lý một mặt thúc đẩy, mặt khác cũng kìm hãm những giới hạn tư duy của các nhà khoa học.\\
\indent
Từ khởi đầu với kích thước lớn như tòa nhà, đến nay kích thước của bóng bán dẫn chỉ còn vài nanomet, công nghệ về máy tính đã phát triển nhanh như vũ bão. Áp lực nâng cao năng lực tính toán, cả về chất và lượng vô hình trung khiến mô hình máy tính hiện tại khó đáp ứng được, do kích thước transitor đã tới giới hạn không thể thu nhỏ, tiệm cận lượng tử. Trong bối cảnh như vậy, cơ học lượng tử đã khai sáng, dẫn lối để các nhà khoa học mở ra hướng đi mới, với hy vọng mở ra kỷ nguyên mới, kỷ nguyên của máy tính lượng tử.\\
\indent
Máy tính lượng tử, với nền tảng từ ba trụ cột khoa học máy tính, toán học và vật lý đã được thai nghén, nghiên cứu và phát triển và bước đầu có những thành tựu nhất định. Nó bắt đầu khai thác một số khía cạnh của cơ học lượng tử để mở rộng sức mạnh cũng như tri thức nhân loại. Nhiều cường quốc trên thế giới cùng các ông lớn công nghệ đang chạy đua để đạt được sức mạnh từ máy tính lượng tử, từ đó giải được những bài toán mà máy tính cổ điển gần như không có cách nào có thể tính toán được.\\
\indent
Mặc dù đã có những thành tựu đầu tiên trong việc tạo ra một máy tính lượng tử, nhưng nó cũng chỉ mới như dấu chân trên sa mạc vạn dặm. Việc nghiên cứu lý thuyết, ứng dụng và chứng minh sức mạnh, thương mại hóa máy tính lượng tử còn cách thành công rất xa. Trong khuôn khổ đề tài, nhóm học viên chỉ đưa ra và trao đổi một cách hết sức sơ khai những tiền đề cơ bản của máy tính lượng tử, không đi sâu vào chi tiết nhằm hiểu rõ chính xác nó. \\
\indent
Phần còn lại của đề tài được bố cục thành 04 phần như sau:
    \begin{itemize}
        \item Chương một giới thiệu tổng quan về máy tính lượng tử. 
        \item Chương hai giới thiệu những điểm cơ bản về kiến trúc của máy tính lượng tử. 
        \item Chương ba mô tả một số thí nghiệm cơ bản chứng minh sự tồn tại và sức mạnh của máy tính lượng tử. 
        \item Chương bốn đưa ra kết luận và hướng phát triển.
    \end{itemize}
\end{preface}
